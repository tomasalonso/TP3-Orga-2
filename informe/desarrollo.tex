\section{Desarrollo}

\subsection{Ejercicio 1}

A lo largo del TP se usan constantes ({\it \#defines\/}), como por ejemplo
en este ejercicio, donde usamos el índice de cada entrada de la {\bf GDT}, declarados
en el archivo {\it defines.h\/}.

Para completar cada entrada se usa la estructura dada por la cátedra sin
ninguna modificación.

% insertar estructura

Se definió un segmento de datos especial para la memoria de video
({\it GDT_VIDEO\/}), el cual se usó solamente en este inciso y se comentó
posteriormente, por ende no se agrega a la GDT.

Luego, se carga la GDT y se salta a modo protegido tal como se vió en las clases
y los ejemplos.

Se cargan todos los segmentos con la modalidad {\bf flat}, todos los segmentos
apuntando a la misma memoria, y se apunta la pila del kernel a la dirección
$0\times27000$, especificada por el enunciado.

Ahora se carga el segmento especial de video y con código assembler se recorre
por medio de 2 ciclos anidados todas las posiciones del segmento que comienzan
en la posición $0\times B8000$, es decir, se recorre cada posición de la
pantalla, pintándola de gris.

De aquí en más se deshabilitó este segmento especial de video, se accede a la
memoria de video por los segmentos de datos y la inicializacion de la pantalla
se realiza en el {\it ejercicio 3\/} con funciones en C.


\subsection{Ejercicio 2}

Para inicializar las entradas de la {\bf IDT} se utilizo la macro {\it IDT_ENTRY\/}
propuesta con una ligera modificacion, se decidió agregar a la misma un
parámetro adicional, que permita especificar el atributo de la entrada,
definidos previamente.
Estas son {\bf TRAP}, la cual indica una trap gate usada para
las excepciones, {\bf INTERRUPT}, usada para interrupciones que solo pueden ser
llamadas por código de privilegio 0, y {\bf USER_INTERRUPT}, que puede ser llamada
por código de usuario también.

\begin{lstlisting}
#define IDT_ENTRY(numero, attribute)
  idt[numero].offset_0_15 = (unsigned short)
                ((unsigned int)(&_isr ## numero) & (unsigned int) 0xFFFF);
  idt[numero].segsel = (unsigned short) (GDT_IDX_ROOT_CODE << 3);
  idt[numero].attr = (unsigned short) attribute;
  idt[numero].offset_16_31 = (unsigned short) (
                (unsigned int)(&_isr ## numero) >> 16 & (unsigned int) 0xFFFF);

#define TRAP      0b1000111100000000
#define INTERRUPT 0b1000111000000000
#define USER_INTERRUPT 0b1110111000000000
\end{lstlisting}

Para la creación de las funciones de las excepciones de procesador se utiliza
otra macro, {\it ISR\/}, esta macro fue adaptada en este ejercicio para que
mostrara un mensaje de error de forma rudimentaria.

\begin{lstlisting}
%macro ISR 2
global _isr%1
  msg%1 db %2, 0
  msg%1_len equ    $ - msg%1
_isr%1:
  mov eax, %1
  imprimir_texto_mp msg%1, msg%1_len, 0x7, 0, 0
  iret
%endmacro
\end{lstlisting}

Cuando se produce una excepción, el descriptor llama a su función asignada
({\it _isrX\/}), la cual, dependiendo del número de interrupción se elige el
mensaje que se imprime por pantalla, esta funcionalidad será aprovechada más
adelante ya que son similares en cuanto a código todas las excepciones en este
tp.

El funcionamiento de la versión final de la macro se detalla mejor en la parte
correspondiente del ejercicio 7, cuando se implementa el verdadero
funcionamiento de las rutinas de atención
de estas excepciones.
% (HABLAR DE LA MACRO DE INTERRUPCIONES)

Una vez inicializada la IDT en el kernel con la macro, como se indica en el
enunciado, se probó a continuación el funcionamiento de varias excepciones.
Esto funcionó correctamente y luego fue comentado para seguir con el trabajo
practico.



\subsection{Ejercicio 3}

En primer lugar se inicializa la pantalla con la función
{\it screen_inicializar\/} hecha en C.

Esta función utiliza todas funciones de la catedra, entre ellas
{\it screen_pintar_rect\/}, {\it screen_pintar_linea_h\/},
{\it screen_pintar_puntajes\/}.

Los pasos a realizar por la función son, en primer lugar, pintar toda la
pantalla de gris; luego se pinta una línea negra en la parte superior, donde se
escribirá informacion, por el momento se escribe aquí el nombre del grupo, pero
luego se escribirá la tecla pulsada y el modo {\it debug\/}.
También se pintan los dos sectores de puntajes de los jugadores y se inicializan
los relojes para cada pirata de cada jugador. Todo lo realizado es acorde a las
imágenes del mapa sugerido por la cátedra.

Luego se implementó {\it mmu_inicializar_dir_kernel\/} que crea un
{\it page directory\/} para el kernel a partir de la dirección $0\times27000$
como indica el enunciado. En este page directory se crea una primera
{\it page table\/} que se ubica en la posición $0\times28000$, posición
asignada directamente y dicho sea de paso, en este punto, al no estar paginación
aún habilitada, pedir memoria libre no es una buena opción.

La {\it page table\/} se cicla con un ciclo y un incremento de 0 a 1024,
asignando páginas sucesivas, de modo que el esquema de paginación sea
{\bf identity mapping}; y para que ocurra efectivamente, se asigna la page table
en la primera entrada del page directory.
De modo que queda mapeada toda la sección del kernel y su memoria libre.
Como última aclaración, la sección del kernel corresponde al uso de la page
table en su totalidad.

Finalmente se carga en el {\bf cr3} del kernel la dirección del page directory y
se habilita en el {\bf cr0} el bit de paginación.


\subsection{Ejercicio 4}

Se iniciliza el seguimiento de las páginas libres con una variable global
({\it proxima_pagina_libre\/}) que comienza indicando el inicio del sector de
memoria libre, ubicado en la posición $0\times10000$. Con esta variable se
indica el lugar donde se reserva la próxima página pedida y luego se mueve una
posición hacia adelante. Este funcionamiento se implementó en la función
{\it mmu_proxima_pagina_fisica_libre\/}.

Luego se implementaron las funciones {\it mmu_mapear_pagina\/} y
{\it mmu_unmapear_pagina\/}, la primera accede al page directory indicado por
el parámetro {\it cr3\/} y chequea si la page table correspondiente a la
direccion virtual que se pide mapear está presente o no; si no lo está, se
reserva una página libre para alojarla, si lo esta, se apunta a la page table
indicada. En esta tabla setea en la entrada correspondiente a la dirección
virtual, los valores necesarios (base) para mapear a la dirección física, que
corresponden a los primeros 20 bits de la dirección, debido a que las páginas
están alineadas a 4kb, y se pone en presente esta la entrada de la página.

Finalmente se limpia el cache de páginas accedidas.

Por otra parte, {\it mmu_unmapear_pagina\/} simplemente setea a cero el bit de
presente de la entrada correspondiente en la page table que estaba mapeando la
dirección virtual indicada.

La implementación de la función {\it mmu_inicializar_dir_pirata\/} es más
compleja y se detalla a continuación:

Se crea un page directory pidiendo una página libre, se setea todo en cero para esta estructura,
en la primera tabla se apunta a la tabla que mapea el kernel, la direccion 0x28000.

Luego se obtiene la direccion fisica en donde comienza el pirata, el puerto del jugador correspondiente y
tambien se localiza con un puntero la posicion en memoria del codigo correspondiente al tipo de pirata.
con el cr3 actual se mapea en la direccion 0x401000 el lugar donde va a ir localizado el codigo,
luego se copia efectivamente el codigo a esta direccion recien mapeada, se ponene manualmente en la pila
los parametros necesarios por el codigo y una direccion de retorno 0 la cual no va a ser llamada nunca y
finalmente se desmapea esta direccion 0x401000.

Se termina por mapear, en la direccion 0x40000 en el page directory creado al comienzo de la funcion, con
la posicion en memoria a donde se copió el codigo de la tarea en los pasos anteriores.

En las 4 tablas de paginas siguientes se mapean 4 tablas las cuales se crean cuando se inicia un jugador.
Estas corresponden a indicar las direcciones del mapa que ya son conocidas por todos los piratas del jugador
y se ven actualizadas con cada movimiento de los piratas exploradores. se hablara de como funcionan especificamente
esta estructura de cuatro paginas al final del informe cuando se detalle el funcionamiento general del juego.   (HABLAR DE TABLAS DE JUGADOR QUE MAPEAN LO DESCUBIERTO)
Por ultimo se da por explorada la posicion en la que se crea el pirata en esta estructura de posiciones exploradas,
en caso de que sea la primera vez, luego se devuelve el page directory del pirata.


\subsection{Ejercicio 5}

Se crean las 3 entradas en la IDT para las 3 interrupciones indicadas.
En la interrupcion del reloj se deshabilitan las interrupciones de PIC
y se realizaba lo pedido de igual manera que lo que recomienda el enunciado, se hablara
en mayor profundida de esta y las otras interrupciones cuando se trate su implementacion final                            (HABLAR DE LAS 3 INTERRUPCIONES AL FINAL)
En la interrupcion del teclado tambien se deshabilitan las interrupciones del PIC y luego se llama a
la funcion game_atender_teclado, la cual en primer lugar solo se encargaba de imprimir en pantalla la tecla
usando un switch y los distintos codigos de teclado, tambien se comentara en la funcion final de esto mas adelante.

Finalmente se agrega la interrupcion 0x46, la cual no hacia nada util y se explica su funcionamiento final más
adelante cuando es correctamente implementada.



\subsection{Ejercicio 6}

Se agrega en la GDT un descriptor de TSS para la tarea incial, uno para la tarea idle, 8 para los piratas de jugador A
y otras 8 para los piratas del jugador B.

En la funcion tss_inicializar se configura la TSS de la tarea idle como se pide, creando esta como una variable
local en el codigo y modificando esta estructura con los valores correspondientes, la pila en la del kernel, el
cr3 del kernel los eflags con valor 0x202 y el eip apuntando a la posicion 0x16000.

Finalmente en un ciclo se inicializan las bases y el bit de presente de todas las TSS de los piratas, que son inicializadas
completamente luego, cuando se lanza un pirata con los valores necesarios de un pirata, pidiendo una pagina para la pila
de nivel 0, seteando su pila al final de su pagina pero asumiendo que tiene los dos paramentros y la direccione
de retorno en la pila y con el eip apuntando al principio de la pagina.

Se escribe el codigo necesario para cargar la tarea inicial como la actual e inmediatamente saltar a la tarea idle.

(ESTO ACA O EN EL EJERCICIO 7?)
Modificamos la interrupcion 0x46 la cual ahora pone en la pila los parametros que se le pasan
y llama a game_syscall_manejar. Esta le pide al scheduler cual es el jugador actual y el pirata actual,
con una serie de ifs chequea cual fue el servicio pedido y llama a game_syscall_pirata_mover, game_syscall_cavar o
game_syscall_pirata_posicion. si no es nungun caso destruye al pirata y devuelve un -1.

Cuando retorna a la interrupcion, esta acomoda la pila, recupera el resultado y chequea si se devolvio un -1,
en este caso salta a la tarea idle, si no, se fija si se intento pedir la posicion, si fue asi, manualmente se
pone en la pila en el registro guardado de la tarea la posicion actual de la misma y tambien se salta a la tarea idle.



\subsection{Ejercicio 7}

La estructura utilizada para el scheduler es la siguiente:
  Se tiene un uint para saber si el scheduler esta activo o no, una lista de uints con 17 posiciones las cuales
  representan los indices de la GDT que llevan al indice de la TSS de la respectiva tarea, las 8 primeras de jugador A,
  las segundas del jugador B y la ultima para la tarea idle.
  Hay un uint que indica el jugador actual (0 para A, 1 para B).

Una tupla de uints que indica en la primera posicion el indice del ultimo pirata ejecutado del Jugador A y en la otra el del
ultimo pirata ejecutado del jugador B en las dos listas siguientes.

En estas dos listas se indica con un int si el pirata correspondiente al indice esta libre (0) o en ejecucion (1);

a) La estructura utilizada para el scheduler es la siguiente:
  Se tiene un uint para saber si el scheduler esta activo o no, una lista de uints con 17 posiciones las cuales
  representan los indices de la GDT que llevan al indice de la TSS de la respectiva tarea, las 8 primeras de jugador A,
  las segundas del jugador B y la ultima para la tarea idle.
  Hay un uint que indica el jugador actual (0 para A, 1 para B).
  Una tupla de uints que indica en la primera posicion el indice del ultimo pirata ejecutado del Jugador A y en la otra el del
  ultimo pirata ejecutado del jugador B en las dos listas siguientes.
  En estas dos listas se indica con un int si el pirata correspondiente al indice esta libre (0) o en ejecucion (1);
b) La implementacion de sched_proxima_a_ejecutar es como sigue:
  en primer lugar se obtiene el jugador activo, el inactivo y se asigna una variable proximo con valor inicial de 16 (la tarea idle)
  Luego, si el scheduler esta activo chequea si el jugador inactivo tiene un slot para ejecutar, esto lo hace con la
  funcion sched_hay_slot_a_ejecutar, en esta funcion se recorre la lista de tareas del jugador indicado y si hay una que esta
  en ejecucion, se devuelve un 1, si no hay ninguna activa, devuelve 0.
  Si esta funcion retorna un 1, se cambia el jugador actual del scheduler por el que se tenia como inactivo y a proximo se le asigna
  el proximo slot de activo de este jugador con la funcion sched_proximo_slot_a_ejecutar, esta funcion recorre el array de slots
  del jugador a partir de la ultima tarea ejecutada (dandola vuelta y volviendo a la misma si no hay otra) y devolviendo el Indices
  de la proxima en la lista que este activa.
  Si el jugador inactivo no tiene un slot a ejecutar, chequea con sched_hay_slot_a_ejecutar lo mismo para el jugador activo y
  tambien asigna a proximo el indice de la proxima tarea de este jugador a ejecutar.
  Si ningun jugador esta activo, devuelve proximo como estaba (indicando la tarea idle).
c) En sched_tick se llama primero a game_tick, luego a game_mineros_pendientes con el jugador actual del scheduler (se habla mas
  de esta funcion luego) y se define el indice de la proxima tarea a ejecutar llamando a sched_proxima_a_ejecutar.
  Finalmente se devuelve el indice de la GDT de la proxima tarea a ejecutar que se obtiene del atributo selectores del scheduler
  con el indice que se consigue en el paso anterior. tambien se modifica el llamado que hace la interrupcion del reloj a game_tick
  por sched_tick.
d) El funcionamiento final de esta instruccion fue detallado en la seccion g) del ejercion 6.
e) Esta modificacion ya se hizo dentro de sched_tick como se explico en la seccion c de este ejercicio.
f) Aqui se modifico la macro que define el comportamiento de cada interrupcion. la unica diferencia entre las interrupciones
  son las que tienen error code y las que no. esto modifica el lugar en la pila de los parametros que se pasan a la funcion de C
  game_atender_excepcion, para solucionar esto, se chequea primero si la excepcion genera error code o no, en base a esto se define un
  offset para acceder a la pila y obtener los datos necesarios para luego ponerlos en el tope de la pila y llamar a la funcion de C.
  La funcion game_atender_excepcion se encarga de llamar a game__pirata_exploto para desalojar el pirata, ademas de chequar
  si se esta en modo debug (con una variable global que se pone en 0 o en 1) para frenar el scheduler, guardar la pantalla actual y mostrar la informacion de debug. de no estar
  en modo debug, vuelve a la interrupcion.
g) cuando game_atender_excepcion entra en modo debug esta hace lo dicho anteriormente, guarda la pantalla actual con screen_guardar,
  imprime lo pedido por el modo debug con screen_debug la cual toma todos los parametros que se pasaron inicialmente por la pila
  el modo debug se inicia cuando se llama a la interrupcion del teclado presionando la tecla y, en este caso se chequea, si ya esta en modo
  debug simplemente se activa el scheduler y se continua la ejecucion. si no, se setea el indicador de debug en 1 lo cual desactivara
  el scheduler cuando haya una excepcion.




game_mineros_pendientes:
  esta funcion esta destinada a chequear si el jugador actual encontro algun botin pero en el momento no tenia lugar para enviar
  un minero.
  primero se chequea si el jugador tiene algun slot libre con sched_hay_slot_libre, si no lo hay se vuelve sin hacer nada,
  si lo hay, se cicla el arreglo botin del jugador, el cual indica con un 1 si el jugador vio ese botin y no lo envio a minar, o
  0 en caso contrario, si se da lo primero se llama a game_jugador_lanzar_minero con destino al botin no minado pero descubierto
  y se setea a cero la posicion correspondiente del arreglo botin del jugadro.

Estructura del mapa explorado:
  Cada jugador posee un arreglo mapa de 4 posiciones, cuanndo se inicializa el jugador se llama a game_jugador_inicializar_mapa,
  esta funcion se encarga de pedir cuatro paginas libres e iniciarlas como page tables vacias, se asigna a las cuatro posiciones
  del arreglo las 4 direcciones de memoria de estas paginas.
  Luego cada pirata tiene su page directory con el kernel y su codigo, se asignan a las cuatro entradas siguientes de este directorio
  las 4 posiciones de memoria de las tablas que se encuentran en el arreglo mapa de su jugador.
  Cuando un pirata se mueve, este tiene su posicion x e y, con estos valores obtiene las posiciones de memoria del mapa a las que se mueve
  y las mapea en este grupo de 4 tablas de paginas, entonces todos los piratas del jugador pueden "ver" lo explorado por los
  otros piratas.

  Estructura Jugador:
      La estructura jugador es la encargada de tener la informacion que comparten todos los piratas asi como
      su puntaje, su puerto y donde se encuentra el codigo de sus tareas.
      La estructura tiene los siguientes atributos:
        Un identificador index (0 para A, 1 para B).
        Un arreglo de piratas cuya capacidad maxima es el total de los que puede tener vivos.
        La cantidad de monedas que tiene recolectadas y su color.
        Un arreglo de 4 posiciones con las direcciones de memoria de las tablas de paginas que se usaran para mapear el mapa visto por
        sus piratas.
        Las posiciones X e Y de su puerto.
        Las dos direcciones a los codigos de sus piratas, minero o explorador.
        un arreglo con tantas posiciones como botines hay en el mapa el cual sirve para indicar cual fue descubierto pero aun no se le envio
        un minero.

  Estructura Pirata:
      La estructura pirata se encarga de conocer a su jugador, su posicion actual y su tipo.
      Esta tiene los siguientes atributos:
        Un index que define que numero de pirata es para su jugador.
        Tiene un puntero a una estructura Jugador, el cual apunta a su jugador.
        Sus posiciones X e Y en el mapa.
        Su tipo (explorador o minero).

  Estructura Scheduler:
      La estructura scheduler es la encargada de saber que jugador esta activo en este momento y tener la informacion
      necesaria para saber que tarea ejecutar en el proximo ciclo del reloj.
      Esta posee los siguientes atributos:
        activo: Indica si el scheduler esta actualmente manejando el juego o no.
        selectores: Este arreglo de 17 posiciones tiene los selectores de las entradas de la GDT de
        las 8 tareas del jugador A, las 8 del B y la tarea idle.
        jugadorActual: Como era de esperarse, indica el jugador actualmente activo.
        slotActual: Es una tupla la cual posee el indice del arreglo slots de la ultima tarea ejecutada por el jugador A
        en su primer valor, y del jugador B en su segundo valor.
        slots: esta tupla de dos arreglos de ocho posiciones indica en cada una de ellas si la tarea correspondiente a ese
        indice (para el jugador A en el primer arreglo y el jugador B en el segundo) esta fuera del juegp (0) o esta actualmente
        en el mapa activa (1).

  Funciones creadas:

    game_minar_botin:
      Esta funcion toma dos parametros que indican una posicion por sus coordenadas X e Y.
      dentro de ella se cicla tantas veces como botines haya y si uno de esos botines esta ubicado en la
      posicion (X,Y) se le resta una moneda al botin y se sale del ciclo.

    La funcion game_lineal2virtual se encarga de pasar de direcciones lineales del mapa a las direcciones
    virtuales correspondientes a la posicion correspondiente a las estructuras de paginacion para el mapa.

    La funcion game_lineal2physical se encarga de pasar de direcciones lineales del mapa a las direcciones
    fisicas correspondientes a la misma posicion pero del mapa en memoria.

    game_actualizar_codigo:
      Esta funcion toma dos puntos (X0,Y0) y (X1,Y1).
      con (X0,Y0) obtiene la direccion virtual para esas coordenadas, que es donde esta el codigo de la tarea ahora
      con (X1,Y1) obtiene la direccion fisica en donde se va a pasar el codigo de la tarea cuando esta se haya movido
      luego se mapea en la direccion 0x400000 la pagina fisica recien obtenida
      para despues copiar desde la posicion actual a la posicion 0x400000, la cual apunta a la nueva posicion de la tarea

    game_calcular_fin:
      Esta funcion chequea si el contador del juego es igual a la variable global FIN, en ese caso se llama a game_terminar_si_es_hora
      y se devuelve un 1.
      Luego si el scheduler esta activo y ambos jugadores tienen todos los slots ocupados y sus puntajes no cambiaron
      (chequeado con dos variables globales puntajeA y puntajeB comparadas con el atributo monedas de cada jugador)
      entonces crece el contador.
      si algo de esto no se cumple, se resetea el contador y se actualizan las variables globales de puntaje de cada jugador.
      finalmente se devuelve un 0.
